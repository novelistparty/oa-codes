% vim:tw=80 sw=2 ft=tex
%         File: pe-start-guide.tex
% Date Created: 2014 Aug 28
%  Last Change: 2014 Aug 28
%     Compiler: pdflatex --shell-escape
%       Author: john
\documentclass[12pt]{article}
\usepackage{amsmath}
\usepackage{graphicx} % for figures
\usepackage{caption}
\usepackage{subcaption} % for mutli-image figures
\usepackage{float} % make figures show up properly

\setlength{\parskip}{1.2ex} % Space between paragraphs
\setlength{\parindent}{0em} % no paragraph indentation
\usepackage[text={6.5in,9in},centering]{geometry} % decrease margins

\usepackage{minted}


% define the title
\author{John Boyle}
\title{Parabolic Equation Implementation Quick-Start Guide}
\begin{document}
\maketitle
\tableofcontents

\section{Introduction}
  This guide will explain the "Recipe for a Simple PE Code" found in
  Chapter 6, Appendix 1 of Computational Ocean Acoustics (COA), 1st
  edition. The PE implementation described here is a variable density
  model, computed with the split-step Fourier method. This guide is
  intended to augment the material in the "Recipe" by bringing together
  the many equations and explanations scattered throughout chapter 6.

\section{PE Code Development}
The recipe is presented below and is followed by a description of
the equations used in each step. Where multiple options were
available, the choice of formulation is explained.

The basic "recipe" that will be followed is:
\flushleft
\begin{enumerate}
\item Select Spatial Grid Size.
\item Create Computational Domain.
\item Add Environment Parameters.
\item Calculate Squared Index of Refraction.
\item Define Starting Field.
\item Choose Marching FFT Algorithm.
\item Choose Numerical Fourier Transform.
\end{enumerate}

\subsection{Select Spatial Grid Size}
For an analytical source function such as the standard Gaussian (Eq.
6.100), an upper bound on the grid size in \textit{z} is 
\begin{equation*}
\Delta z \leq \frac{\lambda}{4}.
\end{equation*}

Grid size in range is given by $ \Delta r = (2 - 5)\Delta z$ for shallow water and 
$ \Delta r = (20 - 50)\Delta z$ for deep water.

\subsection{Create Computational Domain}
The computational domain is created to include the physical domain and
an additional absorption layer to simulate a half-space below the bottom
layer.  It is recommended in COA that the absorption layer thickness (in
depth) be equal to one third the thickness of the physical domain.  The
treatment of absorption in the absorption layer will be explained in the
section "Squared Index of Refraction." 

\subsection{Add Environmental Parameters}
The environmental parameters density, absorption, and sound speed may be
mapped onto matrices of the size of the computational domain.  For
example, the sound speed of a computational ocean of depth D/2 above a
ocean bottom of D/2 could be modeled by creating a matrix of depth D
where the upper half contains the sound speed of water and the lower the
sound speed of rock or sediment.  In a computing environment optimized
for element-wise operations, such as MATLAB, it may be best to create
entire matrices for sound speed, density and absorption.  In other
computer environments and languages, this may be best accomplished
through functions describing the boundaries between features.  In either
case, the results will be directly applied to the calculation of the
squared index of refraction.  For example, sound speed may be described
as follows:

\begin{minted}[mathescape,
               numbersep=5pt,
               gobble=2,
               frame=lines,
               framesep=2mm]{matlab}
  zbot = 500;        % ocean depth
  zmax = 1000;       % max depth of physical domain
  zabs = zmax * 4/3; % absorption layer thickness

  % range and depth vectors
  r = 0 : dr : rmax;
  z = 0 : dz : (zmax + zabs);

  % find indices of zbot and zmax
  zbot_ind = find(z >= zbot, 1);
  zmax_ind = find(z >= zmax, 1);

  % create sound speed environment matrix
  C = ones(length(r),length(z));
  C(:, 1:zbot_ind) = c_water;
  C(:, zbot_ind+1:zmax_ind) = c_sediment;
  C(:, zmax_ind+1:end) = c_absorption; % usually not a constant (see below)
\end{minted}

 
\subsection{Calculate Squared Index of Refraction}
Density, absorption, and sound speed are included in the parabolic
equation, through the squared index of refraction (Eq. 6.145): 
\begin{equation}
	\frac{\partial\psi}{\partial r} = \frac{ik_0}{2} 
		\left( n^2 - 1 + \frac{1}{k_0^2} \frac{\partial^2}{\partial z^2} \right) \psi 
		\label{standard PE}
\end{equation}

Following section 6.5.4 of COA, density variations may be included in
the PE solution by solving the PE for density reduced pressure, 
$\widetilde{\psi} = \psi / \sqrt{\rho}.$ 
This is included in the code through the starting field, 
\[\widetilde{\psi}(0,z) = \psi(0,z) / \sqrt{\rho(0,z)}.\]
Pressure, $\psi(r,z)$, is retrieved from the final numerical field
solution by multiplying by $\sqrt{\rho(r,z)}$.

Density variations are included in the squared index of refraction by
\begin{equation}
	\widetilde{n^2} = n^2 + \frac{1}{2k_0^2}
		\left[
			\frac{1}{\rho(r,z)}\frac{\partial^2 \rho(r,z)}{\partial z^2} - 
			\frac{3}{2\rho(r,z)^2} \left(\frac{\partial\rho(r,z)}{\partial z} \right)^2
		\right]. \label{6.152}
\end{equation}

To avoid infinite derivatives at a density boundary, the function 
\begin{equation}
	\rho(z) = \frac{1}{2}(\rho_1 + \rho_2) + \frac{1}{2}(\rho_2 - \rho_1) 
				\tanh \left(\frac{z - D_0}{L} \right) \label{6.153}
\end{equation}
smooths a boundary between layers of densities $\rho_1$ and $\rho_2$.
The parameter $L$ is chosen with $k_0L = 2$ so that the smoothed
boundary will be an accurate approximation of the real boundary. This
expression may also be used to find an analytic expression for the
derivatives in Eq. 6.151.

In the expression for squared index of refraction (Eq. 6.151), $n^2$
on the right-hand-side is given by 
\begin{equation}
	n^2 \simeq \left(\frac{c_0}{c}\right)^2 
			\left[ 1 + i\frac{\alpha^{(\lambda)}}{27.29} \right],
\end{equation}
where $c$ and $\alpha^{(\lambda)}$, sound speed and absorption in
dB/$\lambda$, are both functions of r and z.

In the absorption layer, squared index of refraction is given by 
\begin{equation}
	n^2 = \left(\frac{c_0}{c_b}\right)^2 + i\alpha \exp \left[ - \left(\frac{z-z_\text{max}}{D}\right)^2 \right].
\end{equation}
where D is the width of the layer and $\alpha$ is generally 0.01. This
gives an exponentially increasing absorption near the computational
boundary and ensures no sound will leave.

\subsection{Define Starting Field}
Choose a source function from Section 6.4.2. 

\subsection{Choose Marching FFT Algorithm}
By introducing the operators
\begin{equation}
A = \frac{ik_0}{2}[n^2(r,z) - 1]; \qquad 
B = \frac{i}{2k_0} \frac{\partial^2}{\partial z^2},
\end{equation} 
the PE \eqref{standard PE} may be written as a first-order differential equation
\begin{equation}
{\partial \psi}{\partial r} = (A + B)\psi = U(r,z)\psi.
\end{equation}
The solution,
\begin{eqnarray}
\psi(r,z) & = & e^{\int_{r_0}^{r_0 + \Delta r} U(r,z)dr} \psi(r_0,z) \nonumber \\
& = & \simeq e^{\widetilde{U}\Delta r} \psi(r_0,z)  
\end{eqnarray}
may be approximated with several different splittings of the exponential
operator $e^{\widetilde{U}\Delta r}. $

The split-step marching solution recommended in the PE recipe in COA is
derived from the exponential operator splitting
	\begin{equation}
		e^{(A + B)\Delta r} \simeq e^{A\Delta r} e^{B\Delta r}.
	\end{equation}
Its error is of order $(\Delta r)^2$.  A different solution,
uses the splitting 
	\begin{equation}
		e^{(A + B)\Delta r} \simeq e^{\frac{A}{2} \Delta r} e^{B \Delta r} e^{\frac{A}{2} \Delta r},
	\end{equation}
whose error is of order $(\Delta r)^3$.  It is given by
	\begin{equation}
		\psi(r,z) = e^{\frac{ik_0}{4}[n^2(r_0,z)-1]\Delta r} \mathcal{F}^{-1} 
			\Big\{
				e^{-\frac{i \Delta r}{2k_0}k_z^2} \mathcal{F} 
				\Big\{
					e^{\frac{ik_0}{4}[n^2(r_0,z)-1]\Delta r} \psi(r_0,z)
				\Big\}
			\Big\} \label{spltstpmarch}
	\end{equation}
	
This formulation is based on the form of the PE called the Tappert
equation.  

\subsection{Choose Numerical Fourier Transform}
The derivation of the PE \eqref{standard PE} assumes $\psi(r,0) = 0$. This boundary
condition may be enforced through the use of the discrete Fast Sine Transform (FST) 
in \eqref{spltstpmarch}. The FST assumes its input is a zero-point odd function, 
therefore, $\psi(r,0) = 0$.  Complex values are easily handled by taking advantage 
of the linearity of Fourier Transforms,  
	\begin{equation}
		\mathcal{F}\big\{x(t) + iy(t)\big\} = \mathcal{F}\big\{x(t)\big\} + 
				i\mathcal{F}\big\{y(t)\big\}.
	\end{equation}

\section{Lloyd Mirror Reference Solution}
A good analytical reference solution is the Lloyd-mirror interference
pattern, where a point source is placed at a depth $z_s$, and an image
source at $-z_s$ is used to enforce the boundary condition of $p(r,0) =
0$.  The equations are
	\begin{equation}
		p(r,z) = \frac{e^{ikR_1}}{R_1} - \frac{e^{ikR_2}}{R_2}
	\end{equation}
where
	\begin{equation}
		R_1 = \sqrt{r^2 + (z - z_s)^2}, \qquad R_2 = \sqrt{r^2 + (z + z_s)^2}
	\end{equation}
The physical environment is a simulated fluid half-space with sound
speed $c = 1500 m/s$.

\section{Conclusion}
  This document has provided a detailed explanation of the PE "Recipe"
  found in COA. Good luck. Please email me if you have questions about
  how to code any of these expressions in MATLAB.

\end{document}
